%% Template for a preprint Letter or Article for submission
%% to the journal Nature.
%% Written by Peter Czoschke, 26 February 2004
%%

\documentclass{nature}
\input{NextDefs.tex}
%% make sure you have the nature.cls and naturemag.bst files where
%% LaTeX can find them

\bibliographystyle{naturemag}

\title{A molecular sensor made of bi-color indicators for single barium atom tagging.}

%% Notice placement of commas and superscripts and use of &
%% in the author list

\author{Aauthor$^{1,2}$, Bauthor$^2$ \& LastAuthor$^2$}


\begin{document}

\maketitle

\begin{affiliations}
 \item Put institutions in this environment and
 \item separate with \verb|\item| commands.
\end{affiliations}

\begin{abstract}
Neutrinos could be Majorana fermions, identical to their own antiparticles.  Establishing where this is the case is one of the most important open question in nuclear and particle physics, and could be achieved observing the lepton-flavor violation process known as neutrinoless double beta decay (\bbonu). After more than forty years of experimentation, the lifetime of such processes in the most interesting isotopes has been found to be longer than $10^{26}$~y, and further progress require increasing the sensitivity by at least one order of magnitude. This, in turn, demand the development of experimental techniques capable of suppressing by up to seventeen orders of magnitude the potential backgrounds induced by Earth's natural radioactivity and cosmic rays.      ``Here we
show...'' or something similar.
\end{abstract}

Articles have a summary, separate from the main text, of up to 150 words, which does not have references, and does not contain numbers, abbreviations, acronyms or measurements unless essential. It is aimed at readers outside the discipline. This summary contains a paragraph (2-3 sentences) of basic-level introduction to the field; a brief account of the background and rationale of the work; a statement of the main conclusions (introduced by the phrase 'Here we show' or its equivalent); and finally, 2-3 sentences putting the main findings into general context so it is clear how the results described in the paper have moved the field forwards.

Articles have a summary, separate from the main text, of up to 150 words, which does not have references, and does not contain numbers, abbreviations, acronyms or measurements unless essential. It is aimed at readers outside the discipline. This summary contains a paragraph (2-3 sentences) of basic-level introduction to the field; a brief account of the background and rationale of the work; a statement of the main conclusions (introduced by the phrase 'Here we show' or its equivalent); and finally, 2-3 sentences putting the main findings into general context so it is clear how the results described in the paper have moved the field forwards.

The nature of neutrino mass is one of the fundamental open questions in nuclear and particle physics.  If neutrinos are Majorana particles, their tiny mass may be evidence for high energy-scale physics via the seesaw mechanism \cite{Chang:1985en,minkowski1977mu,gell1979ramond,yanagida1979proceedings,mohapatra1981neutrino}, and lend support for a compelling theoretical explanation of the matter-antimatter imbalance in the universe (leptogenesis) \cite{Fukugita:1986hr}.  The most sensitive known method to establish the Majorana nature of the neutrino experimentally is direct observation of neutrinoless double beta decay ($0\nu\beta\beta$) \cite{Ostrovskiy:2016uyx,DellOro:2016tmg, GomezCadenas:2010gs,gomez2011search}, a radioactive process that can occur if and only if the neutrino is a Majorana fermion.  The mass scale implied by direct limits \cite{Aseev:2011dq}, cosmology \cite{Ade:2015xua} and neutrino oscillations \cite{Gonzalez-Garcia:2015qrr} dictates that the rate of $0\nu\beta\beta$, assuming the standard mechanism, will be very low: the next generation of experiments must probe $0\nu\beta\beta$ lifetimes of $\geq 10^{27}$ years. Observation of such a rare decay requires ton-scale detectors with near-perfect background rejection capabilities. 

Then the body of the main text appears after the intro paragraph.
Figure environments can be left in place in the document.
\verb|\includegraphics| commands are ignored since Nature wants
the figures sent as separate files and the captions are
automatically moved to the end of the document (they are printed
out with the \verb|\end{document}| command. However, tables must
be manually moved to the end of the document, after the addendum.

Citation of Einstein's paper \cite{Einstein}.

\begin{figure}
%%%\includegraphics{something} % this command will be ignored
\caption{Each figure legend should begin with a brief title for
the whole figure and continue with a short description of each
panel and the symbols used. For contributions with methods
sections, legends should not contain any details of methods, or
exceed 100 words (fewer than 500 words in total for the whole
paper). In contributions without methods sections, legends should
be fewer than 300 words (800 words or fewer in total for the whole
paper).}
\end{figure}

\section*{Another Section}

Sections can only be used in Articles.  Contributions should be
organized in the sequence: title, text, methods, references,
Supplementary Information line (if any), acknowledgements,
interest declaration, corresponding author line, tables, figure
legends.

Spelling must be British English (Oxford English Dictionary)

In addition, a cover letter needs to be written with the
following:
\begin{enumerate}
 \item A 100 word or less summary indicating on scientific grounds
why the paper should be considered for a wide-ranging journal like
\textsl{Nature} instead of a more narrowly focussed journal.
 \item A 100 word or less summary aimed at a non-scientific audience,
written at the level of a national newspaper.  It may be used for
\textsl{Nature}'s press release or other general publicity.
 \item The cover letter should state clearly what is included as the
submission, including number of figures, supporting manuscripts
and any Supplementary Information (specifying number of items and
format).
 \item The cover letter should also state the number of
words of text in the paper; the number of figures and parts of
figures (for example, 4 figures, comprising 16 separate panels in
total); a rough estimate of the desired final size of figures in
terms of number of pages; and a full current postal address,
telephone and fax numbers, and current e-mail address.
\end{enumerate}

See \textsl{Nature}'s website
(\texttt{http://www.nature.com/nature/submit/gta/index.html}) for
complete submission guidelines.

\begin{methods}
Put methods in here.  If you are going to subsection it, use
\verb|\subsection| commands.  Methods section should be less than
800 words and if it is less than 200 words, it can be incorporated
into the main text.

\subsection{Method subsection.}

Here is a description of a specific method used.  Note that the
subsection heading ends with a full stop (period) and that the
command is \verb|\subsection{}| not \verb|\subsection*{}|.

\end{methods}

%% Put the bibliography here, most people will use BiBTeX in
%% which case the environment below should be replaced with
%% the \bibliography{} command.

% \begin{thebibliography}{1}
% \bibitem{dummy} Articles are restricted to 50 references, Letters
% to 30.
% \bibitem{dummyb} No compound references -- only one source per
% reference.
% \end{thebibliography}

\bibliographystyle{naturemag}
\bibliography{sample}


%% Here is the endmatter stuff: Supplementary Info, etc.
%% Use \item's to separate, default label is "Acknowledgements"

\begin{addendum}
 \item Put acknowledgements here.
 \item[Competing Interests] The authors declare that they have no
competing financial interests.
 \item[Correspondence] Correspondence and requests for materials
should be addressed to A.B.C.~(email: myaddress@nowhere.edu).
\end{addendum}

%%
%% TABLES
%%
%% If there are any tables, put them here.
%%

\begin{table}
\centering
\caption{This is a table with scientific results.}
\medskip
\begin{tabular}{ccccc}
\hline
1 & 2 & 3 & 4 & 5\\
\hline
aaa & bbb & ccc & ddd & eee\\
aaaa & bbbb & cccc & dddd & eeee\\
aaaaa & bbbbb & ccccc & ddddd & eeeee\\
aaaaaa & bbbbbb & cccccc & dddddd & eeeeee\\
1.000 & 2.000 & 3.000 & 4.000 & 5.000\\
\hline
\end{tabular}
\end{table}

\end{document}
